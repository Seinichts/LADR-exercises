\section{Self-adjoint and Normal Operators}
\begin{exercise}
  Suppose $n$ is a positive integer. Define $T \in \mathcal{L}(\mathbb{F}^n)$ by \[
    T(z_1, \dots, z_n) = (0, z_1, \dots, z_{n-1}).
  \] Find a formula for $T^*(z_1,\dots,z_n)$
\end{exercise}

\begin{proof}
  By definition, we have
  \begin{equation*}
    \begin{aligned}
      &\langle T(z_1, \dots, z_n), (x_1, \dots ,x_n) \rangle \\
      = &\langle (0, z_1, \dots, z_{n-1}), (z_1, \dots, z_n) \rangle \\
      = &0 \cdot x_1 + z_1 \cdot x_2 + \dots + z_{n-1} \cdot x_n \\
      = &z_1 \cdot 0 + z_1 \cdot x_2 + \dots + z_{n-1} \cdot x_n + z_n \cdot 0 \\
      = &\langle (z_1, \dots, z_n), (x_2, x_3, \dots, 0) \rangle
    \end{aligned}
  \end{equation*}
  Therefore, $T^*(z_1, \dots, z_n) = (z_2, \dots, z_n, 0)$.
\end{proof}

\begin{exercise}
  Suppose $T \in \mathcal{L}(V)$ and $\lambda \in \mathbb{F}$. Prove that
  $\lambda$ is an eigenvalue of $T$ if and only if
  $\overline{\lambda}$ is an eigenvalue of $T^*$.
\end{exercise}

\begin{proof}
  Suppose $\lambda$ is an eigenvalue of $T$, and choose $v \in V$
  be the eigenvector. Then, we have
  \begin{equation*}
    \begin{aligned}
      \langle T(v), v \rangle &= \langle \lambda v, v \rangle\\
      &= \lambda \langle v, v \rangle \\
      &= \langle v, \overline{\lambda} v \rangle \\
      &= \langle v, T^*(v) \rangle
    \end{aligned}
  \end{equation*}
  Therefore, $\lambda$ is an eigenvalue of $T$ if and only if
  $\overline{\lambda}$ is an eigenvalue of $T^*$
\end{proof}

\begin{exercise}
  Suppose $T \in \mathcal{L}(V)$ and $U$ is a subspace of $V$.
  Prove that $U$ is invariant under $T$ if and only if $U^\perp$
  is invariant under $T^*$.
\end{exercise}

\begin{proof}
  Choose $v \in U$ and $w \in U^\perp$,  suppose $U$ is invariant under $T$.
  Then, since $T(v) \in U$, we have \[
    0 = \langle T(v), w \rangle = \langle v, T^*(w) \rangle
  \]. This implies that $T^*(w) \in U^\perp$.
  Therefore, $T^*$ is invariant under $U^\perp$.
\end{proof}

\begin{exercise}
  Suppose $T \in \mathcal{L}(V, W)$. Prove that
  \begin{itemize}[(a)]
    \item $T$ is injective if and only if $T^*$ is surjective;
    \item $T$ is surjective if and only if $T^*$ is injective.
  \end{itemize}
\end{exercise}

\begin{itemize}[(a)]
  \item
    \begin{proof}
      Suppose $T$ is injective.
    \end{proof}
\end{itemize}

\section{The spectral theorem}

\begin{exercise}
  True or false (and give a proof of your answer): There exists $T \in \mathcal{L}(R_b^3)$
  such that $T$ is not self-adjoint (with respect to the usual inner product)
  and such that there is a basis of $\mathbb{R}^3 $ consisting of eigenvectors of $T$
\end{exercise}

\begin{proof}
  The statement is false: suppose there is a basis of $\mathbb{R}^3$ consisting
  of eigenvectors of $T$. Then we could find a diagonal matrix with
  respect the the basis consisting of eigenvectors of $T$. Clearly
  a diagonal matrix equals its transpose. Hence, $T = T^*$, that is,
  $T$ is self-adjoint.
\end{proof}

\begin{exercise}
  Suppose that $T$ is a self-adjoint operator on a finite-dimensional
  inner product space and that 2 and 3 are the only eigenvalues of $T$.
  Prove that $T^2 - 5T + 6I = 0$
\end{exercise}

\begin{proof}
  Suppose $T$ is self-adjoint, then we could find an eigenvector or $T$
  with $\left\| u \right\| = 0$. Let $U = \Span u$, then $U$ and $U^\perp$ is invariant under $T$,
  $T|_U \in L(U)$ and $T|_U \in L(U)$ are self-adjoint. We could choose an
  orthonormal basis in both and the new list created by combining them 
  is also an orthonormal basis consisting of eigenvectors. For any
  $v \in V$, we could decompose v onto the orthonormal basis.
  Then, following from $(T^2 -5T + 6I)v = (T-2I)(T-3I)v$. Since $\lambda \in \{2, 3\}$,
  one of them must be zero. Therefore, $T^2 - 5T + 6I = 0$ as desired.
\end{proof}

\begin{exercise}
  Give an example of an operator $T \in \mathcal{L}(\mathbb{C}^3)$ such that 2 and 3
  are the only eigenvalues of $T$ and $T^2 -5T + 6I \neq 0$.
\end{exercise}

\begin{proof}
  Consider $T$ such that there doesn't exist an orthonormal basis consisting
  of eigenvectors of $T$. Then we could find a $v \in V$ such that v cannot expressed
  as $a_1 \lambda_1 + \dots + a_n \lambda_n$ where $\lambda_n$ are eigenvectors of $T$. Then $T(v)$ does not 
  equals to 0.
\end{proof}

\section{Positive Operators and Isometries}

\begin{exercise}
  Prove or give a counterexaples: If $T \in \mathcal{L}(V)$ is self-adjoint
  and there exists an orthonormal basis $e_1, \dots,e_n $ of $V$ such that 
  $\langle Te_j , e_j \rangle \geq 0$ for each $j$, then $T$ is a positive operator.
\end{exercise}

\begin{proof}
  TODO: Pending
\end{proof}

\begin{exercise}
  Suppose $T$ is a positive operator on $V$. Suppose $v, w \in V$ are such that \[
    Tv = w \text{and} Tw = v \]
  Prove that $v = w$.
\end{exercise}

\begin{proof}
  Suppose $T$ is a positive operator. Then we have
  \begin{equation*}
    \begin{aligned}
      \langle T(v-w), v-w \rangle \geq 0
    \end{aligned}
  \end{equation*}
  On the other hand, we have
  \begin{equation*}
    \begin{aligned}
      \langle T(v-w), v-w \rangle &= \langle Tv - Tw, v-w \rangle \\
      &= - \langle v - w, v-w \rangle \leq 0\\
    \end{aligned}
  \end{equation*}
  Therefore, $\langle v - w, v - w \rangle = 0$ implies $v = w$.
\end{proof}

\begin{exercise}
  Suppose $T$ is a positive operator on $V$ and $U$
  is a subspace of $V$ invariant under $T$. Prove that $T|_U$
  is a positive operator on $U$.
\end{exercise}

\begin{proof}
  Suppose $T$ is a positive operator on $V$ and $U$ is invariant
  under $T$. Then, by definition, for any $u \in U$, $T|_U$ defines a 
  operator in $\mathcal{L}(U)$ and for any $u \in U \subseteq V$, we have \[
    \langle Tu, u \rangle \geq 0
  \]. Therefore, $T|_U$ is a positive operator on $U$.
\end{proof}


\section{Polar Decomposition and Singular Value Decomposition}

\begin{exercise}
	Fix $u, x \in V$ with $u \neq 0$. Define $T \in \mathcal{L}(V)$ by \[
		Tv = \left< v, u \right> x
	\]
	for every $v \in V$. Prove that \[
		\sqrt{T^* T}v = \frac{\left\| x \right\|}{\left\| u \right\|} \langle v , u \rangle u
	\] for every $v \in V$.
\end{exercise}
\begin{proof}
  First we need to find the $T^*$. By definition, we have
	\begin{equation*}
		\begin{aligned}
			\langle v , T^*w \rangle &= \langle Tv , w \rangle \\
			&= \langle \langle v , u \rangle x , w \rangle \\
			&= \langle v , u \rangle \langle x , w \rangle \\
			&= \langle v , \langle w , x \rangle u \rangle \\
      T^* v &= \langle v , x \rangle u
		\end{aligned}
	\end{equation*}
	Then \begin{equation*}
		\begin{aligned}
		 T^*T v &= T^* \langle v , u \rangle x \\
		 &= \langle v , u \rangle T^* x \\
		 &= \langle v , u \rangle \langle x , x \rangle u \\
		 &= \langle v , u \rangle \left\| x \right\|^{2} u
		\end{aligned}
	\end{equation*}
	Notice that the first two terms are scalars,
	the eigenvector could only be $u$ and its eigenvalue
	is $\langle u , u \rangle \left\| x \right\|^{2} = \left\| u \right\|^{2} \left\| x \right\|^{2}$. \\
  Thus, $\sqrt{T^*T}v = \frac{T^*T v}{\sqrt{\lambda}} = \frac{\langle v , u \rangle\left\| x \right\|^{2} u}{\left\| u \right\|\left\| x \right\|}
	= \frac{\left\| x \right\|}{\left\| u \right\|}\langle v , u \rangle u $
\end{proof}

\begin{exercise}
  Find a singular values of the differentiation operator 
  $D \in \mathcal{P}(\mathbb{R}^2)$ defined by $Dp = p'$, where the inner product on
  $\mathcal{P}(\mathbb{R}^2)$ is $\langle p, q \rangle = \int_{-1}^{1} p(x)q(x) \mathrm{d} x$
\end{exercise}

