\documentclass{article}
\sloppy

%%%%%%%%%%%%%%%%%%%%%%%%%%%%%%%%%%%%%%%%%%%%%%%%%%%%%%%%%%%%%%%%%%%%%%
% PACKAGES            																						  %
%%%%%%%%%%%%%%%%%%%%%%%%%%%%%%%%%%%%%%%%%%%%%%%%%%%%%%%%%%%%%%%%%%%%%
\usepackage[dvipsnames]{xcolor}
% \usepackage[10pt]{extsizes}
\usepackage{amsfonts}
\usepackage{amsthm}
\usepackage{amssymb}
\usepackage[shortlabels]{enumitem}
\usepackage{microtype}
\usepackage{amsmath}
\usepackage{mathtools}
\usepackage{commath}
\usepackage{hyperref}

%%%%%%%%%%%%%%%%%%%%%%%%%%%%%%%%%%%%%%%%%%%%%%%%%%%%%%%%%%%%%%%%%%%%%%
% PROBLEM ENVIRONMENT         																			           %
%%%%%%%%%%%%%%%%%%%%%%%%%%%%%%%%%%%%%%%%%%%%%%%%%%%%%%%%%%%%%%%%%%%%%
\usepackage{tcolorbox}
\tcbuselibrary{theorems, breakable, skins}
\newtcbtheorem{prob}% environment name
              {Problem}% Title text
	{% tcolorbox styles
	enhanced,
	colback=Emerald!10,colframe=cyan!40!black,
	fonttitle=\bfseries,
  }%
  {}

\newenvironment{problem}[1]{\begin{prob*}{#1}{}}{\end{prob*}}

%%%%%%%%%%%%%%%%%%%%%%%%%%%%%%%%%%%%%%%%%%%%%%%%%%%%%%%%%%%%%%%%%%%%%%
% THEOREMS/LEMMAS/ETC.         																			  %
%%%%%%%%%%%%%%%%%%%%%%%%%%%%%%%%%%%%%%%%%%%%%%%%%%%%%%%%%%%%%%%%%%%%%%
% \newtheorem{innerproblem}{Problem}
% \newenvironment{problem}[1]{\itshape\renewcommand\theinnerproblem{#1}\innerproblem}{\endinnerproblem}
\newtheorem{thm}{Theorem}
\newtheorem*{thm-non}{Theorem}
\newtheorem{lemma}[thm]{Lemma}
\newtheorem{corollary}[thm]{Corollary}
\newenvironment{solution}{\begin{proof}[Solution]}{\end{proof}}
%%%%%%%%%%%%%%%%%%%%%%%%%%%%%%%%%%%%%%%%%%%%%%%%%%%%%%%%%%%%%%%%%%%%%%
% MY COMMANDS   																						  %
%%%%%%%%%%%%%%%%%%%%%%%%%%%%%%%%%%%%%%%%%%%%%%%%%%%%%%%%%%%%%%%%%%%%%
\newcommand{\Z}{\mathbb{Z}}
\newcommand{\R}{\mathbb{R}}
\newcommand{\C}{\mathbb{C}}
\newcommand{\F}{\mathbb{F}}
\newcommand{\oo}{\infty}
\newcommand{\bigO}{\mathcal{O}}
\newcommand{\Real}{\mathcal{Re}}
\newcommand{\poly}{\mathcal{P}}
\newcommand{\mat}{\mathcal{M}}
\DeclareMathOperator{\Span}{span}
\newcommand{\Hom}{\mathcal{L}}
\DeclareMathOperator{\Null}{null}
\DeclareMathOperator{\Range}{range}
\newcommand{\defeq}{\vcentcolon=}
\newcommand{\restr}[1]{|_{#1}}


%%%%%%%%%%%%%%%%%%%%%%%%%%%%%%%%%%%%%%%%%%%%%%%%%%%%%%%%%%%%%%%%%%%%%%
% SECTION NUMBERING																				           %
%%%%%%%%%%%%%%%%%%%%%%%%%%%%%%%%%%%%%%%%%%%%%%%%%%%%%%%%%%%%%%%%%%%%%%
\renewcommand{\thesection}{\Alph{section}:}




%%%%%%%%%%%%%%%%%%%%%%%%%%%%%%%%%%%%%%%%%%%%%%%%%%%%%%%%%%%%%%%%%%%%%%
% DOCUMENT START              																			           %
%%%%%%%%%%%%%%%%%%%%%%%%%%%%%%%%%%%%%%%%%%%%%%%%%%%%%%%%%%%%%%%%%%%%%%
\title{\vspace{-2em}Chapter 7: Operators on Inner Product Spaces}
\author{Zelong Kuang}
\date{31/05/2024}

\begin{document}
\maketitle

\newpage
\tableofcontents
\newpage

\section{The spectral theorem}

\begin{problem}{1}
  True or false (and give a proof of your answer): There exists $T \in \mathcal{L}(R_b^3)$
  such that $T$ is not self-adjoint (with respect to the usual inner product)
  and such that there is a basis of $\mathbb{R}^3 $ consisting of eigenvectors of $T$
\end{problem}

\begin{proof}
  The statement is false: suppose there is a basis of $\mathbb{R}^3$ consisting
  of eigenvectors of $T$. Then we could find a diagonal matrix with
  respect the the basis consisting of eigenvectors of $T$. Clearly
  a diagonal matrix equals its transpose. Hence, $T = T^*$, that is,
  $T$ is self-adjoint.
\end{proof}

\begin{problem}{2}
  Suppose that $T$ is a self-adjoint operator on a finite-dimensional
  inner product space and that 2 and 3 are the only eigenvalues of $T$.
  Prove that $T^2 - 5T + 6I = 0$
\end{problem}

\begin{proof}
  Suppose $T$ is self-adjoint, then we could find an eigenvector or $T$
  with $\left\| u \right\| = 0$. Let $U = \span u$, then $U$ and $U^\perp$ is invariant under $T$,
  $T|_U \in L(U)$ and $T|_U \in L(U)$ are self-adjoint. We could choose an
  orthonormal basis in both and the new list created by combining them 
  is also an orthonormal basis consisting of eigenvectors. For any
  $v \in V$, we could decompose v onto the orthonormal basis.
  Then, following from $(T^2 -5T + 6I)v = (T-2I)(T-3I)v$. Since $\lambda \in \{2, 3\}$,
  one of them must be zero. Therefore, $T^2 - 5T + 6I = 0$ as desired.
\end{proof}

\begin{problem}{3}
  Give an example of an operator $T \in \mathcal{L}(\mathbb{C}^3)$ such that 2 and 3
  are the only eigenvalues of $T$ and $T^2 -5T + 6I \neq 0$.
\end{problem}

\begin{proof}
  Consider $T$ such that there doesn't exist an orthonormal basis consisting
  of eigenvectors of $T$. Then we could find a $v \in V$ such that v cannot expressed
  as $a_1 \lambda_1 + \dots + a_n \lambda_n$ where $\lambda_n$ are eigenvectors of $T$. Then $T(v)$ does not 
  equals to 0.
\end{proof}

/section{Positive Operators and Isometries}

\begin{problem}{1}
  Prove or give a counterexaples: If $T \in \mathcal{L}(V)$ is self-adjoint
  and there exists an orthonormal basis $e_1, \dots,e_n $ of $V$ such that 
  $\langle Te_j , e_j \rangle \geq 0$ for each $j$, then $T$ is a positive operator.
\end{problem}

\begin{proof}
  TODO: Pending
\end{proof}

\begin{problem}{2}
  Suppose $T$ is a positive operator on $V$. Suppose $v, w \in V$ are such that \[
    Tv = w \text{and} Tw = v \]
  Prove that $v = w$.
\end{problem}

\begin{proof}
  Suppose $T$ is a positive operator. Then we have
  \begin{equation*}
    \begin{aligned}
      \langle T(v-w), v-w \rangle \geq 0
    \end{aligned}
  \end{equation*}
  On the other hand, we have
  \begin{equation*}
    \begin{aligned}
      \langle T(v-w), v-w \rangle &= \langle Tv - Tw, v-w \rangle \\
      &= - \langle v - w, v-w \rangle \leq 0\\
    \end{aligned}
  \end{equation*}
  Therefore, $\langle v - w, v - w \rangle = 0$ implies $v = w$.
\end{proof}

\section{Polar Decomposition and Singular Value Decomposition}

\begin{problem}{1}
	Fix $u, x \in V$ with $u \neq 0$. Define $T \in \mathcal{L}(V)$ by \[
		Tv = \left< v, u \right> x
	\]
	for every $v \in V$. Prove that \[
		\sqrt{T^* T}v = \frac{\left\| x \right\|}{\left\| u \right\|} \langle v , u \rangle u
	\] for every $v \in V$.
\end{problem}
\begin{proof}
  First we need to find the $T^*$. By definition, we have
	\begin{equation*}
		\begin{aligned}
			\langle v , T^*w \rangle &= \langle Tv , w \rangle \\
			&= \langle \langle v , u \rangle x , w \rangle \\
			&= \langle v , u \rangle \langle x , w \rangle \\
			&= \langle v , \langle w , x \rangle u \rangle \\
      T^* v &= \langle v , x \rangle u
		\end{aligned}
	\end{equation*}
	Then \begin{equation*}
		\begin{aligned}
		 T^*T v &= T^* \langle v , u \rangle x \\
		 &= \langle v , u \rangle T^* x \\
		 &= \langle v , u \rangle \langle x , x \rangle u \\
		 &= \langle v , u \rangle \left\| x \right\|^{2} u
		\end{aligned}
	\end{equation*}
	Notice that the first two terms are scalars,
	the eigenvector could only be $u$ and its eigenvalue
	is $\langle u , u \rangle \left\| x \right\|^{2} = \left\| u \right\|^{2} \left\| x \right\|^{2}$. \\
	Thus, $\sqrt{T^*T}v = \frac{T^*T v}{\sqrt{\lambda}} = \frac{\langle v , u \rangle\left\| x \right\|^{2} u}{\left\| u \right\|\left\| x \right\|}
	= \frac{\left\| x \right\|}{\left\| u \right\|}\langle v , u \rangle u $
\end{proof}

\begin{problem}{6}
  Find a singular values of the differentiation operator 
  $D \in \mathcal{P}(\mathbb{R}^2)$ defined by $Dp = p'$, where the inner product on
  $\mathcal{P}(\mathbb{R}^2)$ is $\langle p, q \rangle = \int_{-1}^{1} p(x)q(x) dx$
\end{problem}


% End
\end{document}
