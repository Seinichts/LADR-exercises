\section{Invariant Subspaces}

\begin{exercise}
Suppose $T \in \mathcal{L}(V)$ and $U$ is a subspace of $V$.
\begin{enumerate}[(a)]
	\item Prove that if $U \subset \Null T$, then $U$ is invariant under $T$.
	\item Prove that if $\Range T \subset U$, then $U$ is invariant under $T$.
\end{enumerate}
\end{exercise}
\begin{proof}
	\begin{enumerate}[(a)]
		\item Let $u \in U$. Then $T(u) \in \Null T$ because $u \in \Null T$. Therefore, $T(u) \in U$.
		\item Let $u \in U$. Then $T(u) \in U$ because $T(u) \in \Range T \subset U$.
	\end{enumerate}
\end{proof}

\begin{exercise}
Suppose $S,T \in \mathcal{L}(V)$ are such that $ST = TS$. Prove that $\Null S$ is invariant under $T$.
\end{exercise}

\begin{proof}
	We have $ST(\Null S) = T(S(\Null S)) = T(0) = 0$. This implies that $T(\Null S) \subset \Null S$.
\end{proof}

\begin{exercise}
Suppose $S,T \in \mathcal{L}(V)$ are such that $ST = TS$. Prove that $\Range S$ is invariant under $T$.
\end{exercise}

\begin{proof}
	TODO: 5.A.12 Pending.
	% From $ST = TS$, we have $S(T(\Range S)) = T(S(\Range S))$.
\end{proof}

\begin{exercise}
Suppose that $T \in \mathcal{L}(V)$ and $U_1, \ldots U_m$ are subspaces of $V$ invariant under $T$. Prove that $U_1 + \cdots + U_m$ is invariant under $T$.
\end{exercise}

\begin{proof}
	\begin{align*}
		T(U_1 + \cdots+ U_m) =     & T(U_1) + \cdots + T(U_m) \\
		T(U_1) \subset U_1 \subset & U_1 + \cdots + U_m       \\
		\cdots                                                \\
		T(U_m) \subset U_m \subset & U_1 + \cdots + U_m
	\end{align*}
	Since $U_1 + \cdots + U_m$ is still a subspace, $T(U_1 + \cdots + U_m) \subset U_1 + \cdots + U_m$.
\end{proof}

\newpage

\begin{exercise}
Suppose $T \in \mathcal{L}(V)$. Prove that the intersection of every collection of subspaces of $V$ that are invariant under $T$ is invariant under $T$.
\end{exercise}

\begin{proof}
	Every invariant subspace contains $\{0\}$, and also which is the smallest one. Hence, the intersection is $\{0\}$ and is trivially invariant under $T$.
\end{proof}

\begin{exercise}
Prove or give a counterexample: if $V$ is a finite-dimensional vector space and $U$ is a subspace of $V$ that is invariant under every operator on $V$, then $U = \{0\}$ or $U = V$.
\end{exercise}

\begin{proof}
	Because every operator on $V$ leaves $\{0\}$ invariant, the question turns to prove the existence of an operator under which only $\{0\}$ and $V$ is invariant.

	The case where $\operatorname{dim} V \le 1$ is trivial.
	Suppose $\operatorname{dim} V \ge 2$. We could always construct such an operator $U$. Suppose there exists an invariant subspace $U_1$ of $V$ under an operator $T$ that is neither $\{0\}$ nor $V$ for which $\operatorname{dim} U_1 < \operatorname{dim} V$. Define $T$ which rotates $U_1$ to $W$, where $W \oplus U_1 = V$. $\forall u \in U_1, T(u)$ have some component in $W$, which is not in $U_1$. Thus, $U$ is not invariant under $U$, which is a contradiction. Therefore, $U = \{0\}$ or $U = V$.
\end{proof}

\begin{exercise}
Define $T \in \mathcal{L}(\mathbb{F}^n)$ by \[
	T(x_1,x_2,x_3, \ldots ,x_n) = (x_1,2x_2,3x_3, \ldots ,n x_n)\]
\begin{enumerate}[(a)]
	\item Find all Eigenvalues and eigenvectors of $T$.
	\item Find all invariant subspaces of $T$.
\end{enumerate}
\end{exercise}
\begin{proof}
	\begin{enumerate}[(a)]
		\item
		      \begin{proof}
			      the eigenvalues and the corresponding eigenvectors are i and $(0, \ldots ,x_i, \ldots ,0)$
		      \end{proof}
		\item \begin{proof}
			      The invariant subspaces are $\{0\}$, $\mathbb{F}^n$, and the subspaces spanned by the eigenvectors.
		      \end{proof}
	\end{enumerate}
\end{proof}

\newpage
\begin{exercise}
Define $T \in \mathcal{L}(\mathcal{P}_4(\mathbb{R}))$ by \[
	(Tp)(x) = x p'(x)\]
for all $x \in \mathbb{R}$. Find all eigenvalues and eigenvectors of $T$.
\end{exercise}
\begin{proof}
	\begin{align*}
		\lambda p(x) & = Tp(x)                            \\
		\lambda a_4x^4 + \lambda a_3x^{3} + \lambda a_2x^{2} + \lambda a_1 x + \lambda a_0
		             & = x (4a_4  x^3+3a_3x^2+2a_2 x+a_1) \\
		             & = 4a_4 x^4+3a_3x^3+2a_2 x^2+a_1 x
	\end{align*}
	The eigenvalues and eigenvectors are $i$ and $ix^i$ respectively.
\end{proof}

\begin{exercise}
Suppose $V$ is finite-dimensional, $T \in \mathcal{L}(V)$, and $\lambda \in F$. Prove that there exists $\alpha \in \mathbb{F}$ such that $\left\vert \alpha - \lambda \right\vert < \frac{1}{1000}$ and $T - \alpha I$ is invertible.
\end{exercise}
\begin{proof}
	We only need to make $\alpha$ not be an eigenvalue of $T$. We could achieve this by the following procedure:
	suppose $\lambda$ is an eigenvalue of $T$, then $T - \lambda I$ is not invertible. We could then choose $\alpha = \lambda + \frac{1}{1000 + i}$, where $i \in \left\{1, \ldots ,\operatorname{dim}V + 1\right\}$, which is not an eigenvalue of $T$.
\end{proof}

\begin{exercise}
Suppose $V = U \oplus W$, where $U$ and $W$ are nonzero subspaces of $V$. Define $P \in \mathcal{L}(V)$ by $P(u + w) = u$ for $u \in U$ and $w \in W$. Find all eigenvalues and eigenvectors of $P$.
\end{exercise}
\begin{proof}
	The eigenvalues are 1 and 0, and the eigenvectors are $u$ and $w$ respectively.
\end{proof}

\begin{exercise}
Suppose $T \in \mathcal{L}(V)$. Suppose $S \in \mathcal{L}(V)$ is invertible.
\begin{enumerate}[(a)]
	\item Prove that $T$ and $S^{-1}TS$ have the same eigenvalues.
	\item What is the relationship between the eigenvectors of $T$ and those of $S^{-1}TS$?
\end{enumerate}
\end{exercise}
\begin{enumerate}[(a)]
	\item \begin{proof}
		      Let $\lambda$ be a eigenvalue of $T$ and $v$ be the corresponding eigenvector. Then $T(v) = \lambda v$. Now we will verify that $\lambda$ is also an eigenvalue of $S^{-1}TS$. \par
		      $S^{-1}(T)S(S^{-1}(v)) = S^{-1}(\lambda v) = \lambda S^{-1}(v)$ since $S$ is invertible. Therefore, eigenvalue for $T$ is also an eigenvalue of $S^{-1}TS$. Similarly, let $\lambda$ be an eigenvalue of $S^{-1}TS$ and $v$ be the corresponding eigenvector such that $S^{-1}TS(v) = \lambda v$. Notice that $S(S^{-1}TS)S^{-1} = T$. Hence, an eigenvalue for $S^{-1}TS$ is also an eigen value for $T$. Therefore, the eigenvalues are the same.
	      \end{proof}
	\item
	      \begin{proof}
		      The eigenvectors of $S^{-1}TS$ are $S^{-1}v$.
	      \end{proof}
\end{enumerate}

\begin{exercise}
Suppose $V$ is a complex vector space, $T \in \mathcal{L}(V)$, and the matrix of $T$ with respect to some basis of $V$ contains only real entries. Show that if $\lambda$ is an eigenvalue of $T$, then so is $\bar{\lambda}$.
\end{exercise}
\begin{proof}
	Suppose $\lambda$ is an eigenvalue of $T$,
\end{proof}

\begin{exercise}
Show that the operator $T \in \mathcal{L}(\mathbb{C}^\infty)$ defined by \[
	T(z_1,z_2, \ldots ) = (0, z_1, z_2, \ldots )\]
has no eigenvalues.
\end{exercise}
\begin{proof}
	Suppose $\lambda$ is an eigenvalue of $T$, and $(0, z_1, z_2, \ldots )$ be the corresponding eigenvector. Then $T(z_1,z_2, \ldots ) = \lambda (0, z_1, z_2, \ldots )$. This implies that $z_1 = 0$, and $z_2 = \lambda z_1 = 0$, and so on. Therefore, the eigenvector is $(0,0,0, \ldots )$, which is not an eigenvector.
\end{proof}

\begin{exercise}
Suppose $n$ is a positive integer and $T \in \mathcal{L}(\mathbb{F}^n)$ is defined by \[
	T(x_1, \ldots ,x_n) = (x_1 + \cdots + x_n, \ldots , x_1 + \cdots + x_n)\]
in other words, $T$ is the operator whose matrix (with respect to the standard basis) consists of all 1's. Find all eigenvalues and eigenvectors of $T$.
\end{exercise}
\begin{proof}
	The eigenvalues are $n$ and $0$, and the eigenvectors are $(1, \ldots ,1)$, and $\left\{(x_1, \ldots ,x_n) \in \mathbb{F}^n / \left\{0\right\} : x_1 + \cdots +x_n = 0 \right\} $ respectively.
\end{proof}

TODO: Sec.A 20 and beyond

\section{Eigenvectors and Upper-Triangular Matrices}

\begin{exercise}
Suppose $T \in \mathcal{L}(V)$ and there exists a positive integer $n$ such that $T^{n} = 0$
\begin{enumerate}[(a)]
	\item Prove that $I - T$ is invertible and that \[
		      (I - T)^{-1} = I + T + \cdots + T^{n-1}.\]
	\item Explain how you would guess the formula above.
\end{enumerate}
\end{exercise}
\begin{enumerate}[(a)]
	\item \begin{proof}
		      Notice that \[
			      (I - T)(I + T + \cdots + T^{n - 1}) = (I + T + \cdots + T^{n - 1})(I - T) = I - T^n = I\]
	      \end{proof}
	\item \begin{proof}
	      \end{proof}
\end{enumerate}

\begin{exercise}
Suppose $T \in \mathcal{L}(V)$ and $(T-2I)(T-3I)(T-4I) = 0$. Suppose $\lambda$ is an eigenvalue of $T$. Prove that $\lambda = 2$ or $\lambda = 3$ or $\lambda = 4$
\end{exercise}
\begin{proof}
	Suppose $(T-2I)(T-3I)(T-4I) = 0$, this implies that $T$ is upper triangular. Therefore, the eigenvalues are either 2, 3, or 4.
\end{proof}

\begin{exercise}
Suppose $T \in \mathcal{L}(V)$ and $T^{2} = I$ and -1 is not an eigenvalue of $T$. Prove that $T = I$.
\end{exercise}
\begin{proof}
	Suppose $T^{2} = I$. Following from $T^{2}* T = T$, we have $T^{2n} = I$. Now if $T \neq I$, then a matrix of $T$ with respect to some basis which is upper triangular has eigenvalues 1 and -1. This is a contradiction.
\end{proof}

\begin{exercise}
Suppose $P \in \mathcal{L}(V)$ and $P^{2} = P$. Prove that $V = \Null P \oplus \operatorname{Range} P$.
\end{exercise}
\begin{proof}
	$P^{2} = P$ implies that $P$ is invariant under $V$. Therefore, $V = \Null P \oplus \operatorname{Range} P$.
\end{proof}

\begin{exercise}
Suppose $S,T \in \mathcal{L}(V)$ and $S$ is
invertible. Suppose $p \in \mathcal{P}(\mathbb{F})$
is a polynomial. Prove that\[
	p(STS^{-1}) = Sp(T)S^{-1}\]
\end{exercise}
\begin{proof}
	From the properties of polynomials, we have
	\begin{equation*}
		\begin{aligned}
			p(STS^{-1}) & = a_0I + a_1STS^{-1} + \cdots + a_n(STS^{-1})^{n}                   \\
			            & = a_0I + a_1STS^{-1} + \cdots + a_nSTS^{-1}STS^{-1} \cdots STS^{-1} \\
			            & = a_0I + a_1STS^{-1} + \cdots + a_nST^{n}S^{-1}                     \\
			            & = Sp(T)S^{-1}
		\end{aligned}
	\end{equation*}
\end{proof}

\begin{exercise}
Suppose $T \in \mathcal{L}(V)$ and $U$ is a subspace of $V$ invariant under $T$. Prove that $U$ is invariant under $p(T)$ for every polynomial $p \in \mathcal{P}(\mathbb{F})$
\end{exercise}
\begin{proof}
	From the properties of polynomials, we have
	\begin{equation*}
		\begin{aligned}
			p(T)(U) & = a_0I + a_1T + \cdots + a_nT^{n}(U)       \\
			        & = a_0I(U) + a_1T(U) + \cdots + a_nT^{n}(U) \\
			        & = U
		\end{aligned}
	\end{equation*}
\end{proof}

\begin{exercise}
Suppose $T \in \mathcal{L}(V)$. Prove that 9 is an eigenvalue of $T^{2}$ if and only if 3 or -3 is an
eigenvalue of $T$.
\end{exercise}
\begin{proof}
	The upper triangle matrix of $T^{2}$ with respect to
	some basis has $9$ on the diagonal, namely eigenvalue,
	therefore, the eigenvalues of $T$ are $\sqrt{9} = \pm 3$
\end{proof}

\begin{exercise}
Give an example of $T \in \mathcal{L}(\mathbb{R}^{2})$
such that $T^{4} = -1$.
\end{exercise}
\begin{proof}
	\begin{comment}
	TODO: 5.B.8 Pending. Give an example.
	\end{comment}
\end{proof}

\newpage
\begin{exercise}
Suppose $V$ is finite-dimensional, $T \in \mathcal{L}(V)$, and $v \in V$ with $v \neq  0$. Let $p$ be a nonzero polynomial of smallest degree such that $p(T)v = 0$. Prove that every zero of $p$ is an eigenvalue of $T$.
\end{exercise}
\begin{proof}
	Suppose $\lambda \in \mathbb{F}$ is a zero of $p$. Then by the fundamental theorem of algebra, we have $p(x) = (x - \lambda)q(x)$, where $q(x)$ is a polynomial of degree $n - 1$. Therefore, $p(T)v = (T - \lambda I)q(T)v = 0$. Since $p$ is of smallest degree, $q(T)v \neq 0$. Hence, $\lambda$ is an eigenvalue of $T$.
\end{proof}

\begin{exercise}
Suppose $T \in \mathcal{L}(V)$ and $v$ is an eigenvetor of $T$ with eigenvalue $\lambda$. Suppose $p \in \mathcal{P}(\mathbb{F})$. Prove that $p(T)v = p(\lambda)v$
\end{exercise}
\begin{proof}
	\begin{equation*}
		\begin{aligned}
			p(T)v & = a_0v + a_1Tv + \cdots + a_nT^{n}v              \\
			      & = a_0v + a_1\lambda v + \cdots + a_n\lambda^{n}v \\
			      & = p(\lambda)v
		\end{aligned}
	\end{equation*}
\end{proof}

\begin{exercise}
Suppose $\mathbb{F} = \mathbb{C}$, $T \in \mathcal{L}(V)$, $p \in \mathcal{P}(\mathbb{C})$ is a polynomial, and $\alpha \in \mathbb{C}$. Prove that $\alpha$ is an eigenvalue of $p(T)$ if and only if $\alpha = p(\lambda)$ for some eigenvalue $\lambda$ of $T$.
\end{exercise}
\begin{proof}
	Suppose $\alpha$ is an eigenvalue of $p(T)$,
	we have $p(T) = (T-\lambda I)q(T)$.
\end{proof}

\section{Eigenspaces and Diagonal Matrices}

\begin{exercise}
Suppose $T \in \mathcal{L}(V)$ is diagonalizable. Prove that $V = \Null T \oplus \operatorname{Range} T$.
\end{exercise}
\begin{proof}
	Since $T$ is diagonalizable, $V$ has a basis of eigenvectors of $T$.
	Therefore, $V = \Null T \oplus \operatorname{Range} T$.
\end{proof}

\begin{exercise}
Prove the converse of the statement in the exercise above
or give a counterexample to the converse.
\end{exercise}

\begin{proof}
	% TODO: 5.C.2 Pending. Give a counterexample.
\end{proof}

\begin{exercise}
Suppose $V$ is finite-dimensional and $T \in \mathcal{L}(V)$.
Prove that the following are equivalent:
\begin{enumerate}[(a)]
	\item $V = \Null T \oplus \Range T$
	\item $V = \Null T + \Range T$
	\item $\Null T \cap \Range T = \{0\}$
\end{enumerate}
\end{exercise}
\begin{proof}
	(a) $\iff$ (b): This is trivial. \\
	(b) $\to$ (c): Suppose $V = \Null T + \Range T$. By Theorem, we have
	\begin{align*}
		\dim V & = \dim (\Null T + \Range T)                                     \\
		       & = \dim \Null T + \dim \Range T + \dim ( \Null T \cap \Range T )
	\end{align*}
\end{proof}

\begin{exercise}
Give an example to show that the exercise above is false without
the hypothesis that $V$ is finite-dimensional.
\end{exercise}

\begin{proof}
	% TODO  5.C.4 Pending. Give an example.
\end{proof}

\begin{exercise}
Suppose $V$ is a 1finite-dimensional complex vector space and
$T \in \mathcal{L}(V)$. Prove that $T$ is diagonalizable if and only if \[
	V = \Null (T - \lambda I) \oplus \Range (T - \lambda I)\]
for every $\lambda \in \mathbb{C}$
\end{exercise}
\begin{proof}
	Suppose $T$ is diagonalizable, the eigenvectors of
	$T$ form a basis of $V$. This impiles that $T - \lambda I$
	has same dimension to $V$. That is, $V = \Null (T - \lambda I) \oplus \Range (T - \lambda I)$.
\end{proof}

\begin{exercise}
Suppose $V$ is finite-dimensional, $T \in \mathcal{L}(V)$ has $\dim V$ distinct
eigenvalues, and $S \in \mathcal{L}(V)$ has the same eigenvectors as $T$
(not necessarily with the same eigenvalues). Prove that $ST = TS$.
\end{exercise}
\begin{proof}
	Since $S$ and $T$ have the same eigenvectors.
	Then $S$ and $T$ are diagonalizable, and this implies that $ST = TS$.
\end{proof}

\begin{exercise}
Suppose $T \in \mathcal{L}(\mathbb{F}^5)$ and $\dim E(8,T) = 4$.
Prove that $T - 2I$ or $T - 6I$ is invertible.
\end{exercise}
\begin{proof}
	Suppose $\dim E(8,T) = 4$, then we have 4 independent
	eigenvectors of $T$ with eigenvalue 8. And the diagonal matrix will have 4 8's on the diagonal.
	Therefore, Suppose $T - 2I$ or $T - 6I$ is not invertible, that is, 2 or 6 is an eigenvalue of $T$.
	the upper-triangular matrix of $T$ will have 2 or 6 on the diagonal since $\dim T \leq 5$, which is a contradiction.
\end{proof}

\begin{exercise}
Suppose $T \in \mathcal{L}(V)$ is invertible. Prove that
$E(\lambda T) = E(\frac{1}{\lambda}, T^{-1})$ for every $\lambda \in \mathbb{F}$
with $\lambda \neq 0$.
\end{exercise}

