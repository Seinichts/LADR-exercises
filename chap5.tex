\documentclass{article}
\sloppy

%%%%%%%%%%%%%%%%%%%%%%%%%%%%%%%%%%%%%%%%%%%%%%%%%%%%%%%%%%%%%%%%%%%%%%
% PACKAGES            																						  %
%%%%%%%%%%%%%%%%%%%%%%%%%%%%%%%%%%%%%%%%%%%%%%%%%%%%%%%%%%%%%%%%%%%%%
\usepackage{xeCJK}
\usepackage[dvipsnames]{xcolor}
\usepackage[10pt]{extsizes}
\usepackage{amsfonts}
\usepackage{amsthm}
\usepackage{amssymb}
\usepackage[shortlabels]{enumitem}
\usepackage{microtype}
\usepackage{amsmath}
\usepackage{mathtools}
\usepackage{commath}
\usepackage{hyperref}

\setCJKmainfont{Noto Sans CJK SC}

%%%%%%%%%%%%%%%%%%%%%%%%%%%%%%%%%%%%%%%%%%%%%%%%%%%%%%%%%%%%%%%%%%%%%%
% PROBLEM ENVIRONMENT         																			           %
%%%%%%%%%%%%%%%%%%%%%%%%%%%%%%%%%%%%%%%%%%%%%%%%%%%%%%%%%%%%%%%%%%%%%
\usepackage{tcolorbox}
\tcbuselibrary{theorems, breakable, skins}
\newtcbtheorem{prob}% environment name
              {Problem}% Title text
	{% tcolorbox styles
	enhanced,
	colback=Emerald!10,colframe=cyan!40!black,
	fonttitle=\bfseries,
  }%
  {}

\newenvironment{problem}[1]{\begin{prob*}{#1}{}}{\end{prob*}}

%%%%%%%%%%%%%%%%%%%%%%%%%%%%%%%%%%%%%%%%%%%%%%%%%%%%%%%%%%%%%%%%%%%%%%
% THEOREMS/LEMMAS/ETC.         																			  %
%%%%%%%%%%%%%%%%%%%%%%%%%%%%%%%%%%%%%%%%%%%%%%%%%%%%%%%%%%%%%%%%%%%%%%
% \newtheorem{innerproblem}{Problem}
% \newenvironment{problem}[1]{\itshape\renewcommand\theinnerproblem{#1}\innerproblem}{\endinnerproblem}
\newtheorem{thm}{Theorem}
\newtheorem*{thm-non}{Theorem}
\newtheorem{lemma}[thm]{Lemma}
\newtheorem{corollary}[thm]{Corollary}
\newenvironment{solution}{\begin{proof}[Solution]}{\end{proof}}
%%%%%%%%%%%%%%%%%%%%%%%%%%%%%%%%%%%%%%%%%%%%%%%%%%%%%%%%%%%%%%%%%%%%%%
% MY COMMANDS   																						  %
%%%%%%%%%%%%%%%%%%%%%%%%%%%%%%%%%%%%%%%%%%%%%%%%%%%%%%%%%%%%%%%%%%%%%
\newcommand{\Z}{\mathbb{Z}}
\newcommand{\R}{\mathbb{R}}
\newcommand{\C}{\mathbb{C}}
\newcommand{\F}{\mathbb{F}}
\newcommand{\oo}{\infty}
\newcommand{\bigO}{\mathcal{O}}
\newcommand{\Real}{\mathcal{Re}}
\newcommand{\poly}{\mathcal{P}}
\newcommand{\mat}{\mathcal{M}}
\DeclareMathOperator{\Span}{span}
\newcommand{\Hom}{\mathcal{L}}
\DeclareMathOperator{\Null}{null}
\DeclareMathOperator{\Range}{range}
\newcommand{\defeq}{\vcentcolon=}
\newcommand{\restr}[1]{|_{#1}}


%%%%%%%%%%%%%%%%%%%%%%%%%%%%%%%%%%%%%%%%%%%%%%%%%%%%%%%%%%%%%%%%%%%%%%
% SECTION NUMBERING																				           %
%%%%%%%%%%%%%%%%%%%%%%%%%%%%%%%%%%%%%%%%%%%%%%%%%%%%%%%%%%%%%%%%%%%%%%
\renewcommand{\thesection}{\Alph{section}:}



%%%%%%%%%%%%%%%%%%%%%%%%%%%%%%%%%%%%%%%%%%%%%%%%%%%%%%%%%%%%%%%%%%%%%%
% DOCUMENT START              																			           %
%%%%%%%%%%%%%%%%%%%%%%%%%%%%%%%%%%%%%%%%%%%%%%%%%%%%%%%%%%%%%%%%%%%%%%
\title{\vspace{-2em}Chapter 5: Eigenvalues, Eigenvectors, and invariant subspaces}
\author{Zelong Kuang}

\begin{document}
\maketitle

\newpage
\tableofcontents
\newpage

\section{Invariant Subspaces}

\begin{problem}{1}
Suppose $T \in \mathcal{L}(V)$ and $U$ is a subspace of $V$.
\begin{enumerate}[(a)]
  \item Prove that if $U \subset \Null T$, then $U$ is invariant under $T$.
  \item Prove that if $\Range T \subset U$, then $U$ is invariant under $T$.
\end{enumerate}
\end{problem}
\begin{proof}
  \begin{enumerate}[(a)]
    \item Let $u \in U$. Then $T(u) \in \Null T$ because $u \in \Null T$. Therefore, $T(u) \in U$.
    \item Let $u \in U$. Then $T(u) \in U$ because $T(u) \in \Range T \subset U$.
  \end{enumerate}
\end{proof}

\begin{problem}{2}
Suppose $S,T \in \mathcal{L}(V)$ are such that $ST = TS$. Prove that $\Null S$ is invariant under $T$.
\end{problem}

\begin{proof}
We have $ST(\Null S) = T(S(\Null S)) = T(0) = 0$. This implies that $T(\Null S) \subset \Null S$.
\end{proof}

\begin{problem}{3}
Suppose $S,T \in \mathcal{L}(V)$ are such that $ST = TS$. Prove that $\Range S$ is invariant under $T$.
\end{problem}

\begin{proof}
TODO: 5.A.12 Pending.
% From $ST = TS$, we have $S(T(\Range S)) = T(S(\Range S))$.
\end{proof}

\begin{problem}{4}
Suppose that $T \in \mathcal{L}(V)$ and $U_1, \ldots U_m$ are subspaces of $V$ invariant under $T$. Prove that $U_1 + \cdots + U_m$ is invariant under $T$.
\end{problem}

\begin{proof}
\begin{align*}
T(U_1 + \cdots+ U_m) =& T(U_1) + \cdots + T(U_m) \\
T(U_1) \subset U_1 \subset& U_1 + \cdots + U_m\\
\cdots \\
T(U_m) \subset U_m \subset& U_1 + \cdots + U_m
\end{align*}
Since $U_1 + \cdots + U_m$ is still a subspace, $T(U_1 + \cdots + U_m) \subset U_1 + \cdots + U_m$.
\end{proof}

\newpage

\begin{problem}{5}
Suppose $T \in \mathcal{L}(V)$. Prove that the intersection of every collection of subspaces of $V$ that are invariant under $T$ is invariant under $T$.
\end{problem}

\begin{proof}
Every invariant subspace contains $\{0\}$, and also which is the smallest one. Hence, the intersection is $\{0\}$ and is trivially invariant under $T$.
\end{proof}

\begin{problem}{6}
Prove or give a counterexample: if $V$ is a finite-dimensional vector space and $U$ is a subspace of $V$ that is invariant under every operator on $V$, then $U = \{0\}$ or $U = V$.
\end{problem}

\begin{proof}
  Because every operator on $V$ leaves $\{0\}$ invariant, the question turns to prove the existence of an operator under which only $\{0\}$ and $V$ is invariant.

  The case where $\operatorname{dim} V \le 1$ is trivial.
  Suppose $\operatorname{dim} V \ge 2$. We could always construct such an operator $U$. Suppose there exists an invariant subspace $U_1$ of $V$ under an operator $T$ that is neither $\{0\}$ nor $V$ for which $\operatorname{dim} U_1 < \operatorname{dim} V$. Define $T$ which rotates $U_1$ to $W$, where $W \oplus U_1 = V$. $\forall u \in U_1, T(u)$ have some component in $W$, which is not in $U_1$. Thus, $U$ is not invariant under $U$, which is a contradiction. Therefore, $U = \{0\}$ or $U = V$.
\end{proof}

\begin{problem}{8}
Define $T in \mathcal{L}(\mathbb{F}^2)$ by \[
  T(w,z) = (z,w).\]
Find all eigenvalues and eigenvectors of $T$.
\end{problem}

% End
\end{document}
