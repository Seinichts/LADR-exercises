\documentclass{extarticle}
\sloppy

%%%%%%%%%%%%%%%%%%%%%%%%%%%%%%%%%%%%%%%%%%%%%%%%%%%%%%%%%%%%%%%%%%%%%%
% PACKAGES            																						  %
%%%%%%%%%%%%%%%%%%%%%%%%%%%%%%%%%%%%%%%%%%%%%%%%%%%%%%%%%%%%%%%%%%%%%
\usepackage[10pt]{extsizes}
\usepackage{amsfonts}
\usepackage{amsthm}
\usepackage{amssymb}
\usepackage[shortlabels]{enumitem}
\usepackage{microtype}
\usepackage{amsmath}
\usepackage{mathtools}
\usepackage{commath}
\usepackage{hyperref}

%%%%%%%%%%%%%%%%%%%%%%%%%%%%%%%%%%%%%%%%%%%%%%%%%%%%%%%%%%%%%%%%%%%%%%
% PROBLEM ENVIRONMENT         																			           %
%%%%%%%%%%%%%%%%%%%%%%%%%%%%%%%%%%%%%%%%%%%%%%%%%%%%%%%%%%%%%%%%%%%%%
\usepackage{tcolorbox}
\tcbuselibrary{theorems, breakable, skins}
\newtcbtheorem{prob}% environment name
              {Problem}% Title text
  {enhanced, % tcolorbox styles
  attach boxed title to top left={xshift = 4mm, yshift=-2mm},
  colback=blue!5, colframe=black, colbacktitle=blue!3, coltitle=black,
  boxed title style={size=small,colframe=gray},
  fonttitle=\bfseries,
  separator sign none
  }%
  {} 
\newenvironment{problem}[1]{\begin{prob*}{#1}{}}{\end{prob*}}

%%%%%%%%%%%%%%%%%%%%%%%%%%%%%%%%%%%%%%%%%%%%%%%%%%%%%%%%%%%%%%%%%%%%%%
% THEOREMS/LEMMAS/ETC.         																			  %
%%%%%%%%%%%%%%%%%%%%%%%%%%%%%%%%%%%%%%%%%%%%%%%%%%%%%%%%%%%%%%%%%%%%%%
\newtheorem{thm}{Theorem}
\newtheorem*{thm-non}{Theorem}
\newtheorem{lemma}[thm]{Lemma}
\newtheorem{corollary}[thm]{Corollary}

%%%%%%%%%%%%%%%%%%%%%%%%%%%%%%%%%%%%%%%%%%%%%%%%%%%%%%%%%%%%%%%%%%%%%%
% MY COMMANDS   																						  %
%%%%%%%%%%%%%%%%%%%%%%%%%%%%%%%%%%%%%%%%%%%%%%%%%%%%%%%%%%%%%%%%%%%%%
\newcommand{\Z}{\mathbb{Z}}
\newcommand{\R}{\mathbb{R}}
\newcommand{\C}{\mathbb{C}}
\newcommand{\F}{\mathbb{F}}
\newcommand{\oo}{\infty}
\newcommand{\bigO}{\mathcal{O}}
\newcommand{\Real}{\mathcal{Re}}
\newcommand{\poly}{\mathcal{P}}
\newcommand{\mat}{\mathcal{M}}
\DeclareMathOperator{\Span}{span}
\newcommand{\Hom}{\mathcal{L}}
\DeclareMathOperator{\Null}{null}
\DeclareMathOperator{\Range}{range}
\newcommand{\defeq}{\vcentcolon=}
\newcommand{\restr}[1]{|_{#1}}


%%%%%%%%%%%%%%%%%%%%%%%%%%%%%%%%%%%%%%%%%%%%%%%%%%%%%%%%%%%%%%%%%%%%%%
% SECTION NUMBERING																				           %
%%%%%%%%%%%%%%%%%%%%%%%%%%%%%%%%%%%%%%%%%%%%%%%%%%%%%%%%%%%%%%%%%%%%%%
\renewcommand\thesection{\Alph{section}:}



%%%%%%%%%%%%%%%%%%%%%%%%%%%%%%%%%%%%%%%%%%%%%%%%%%%%%%%%%%%%%%%%%%%%%%
% DOCUMENT START              																			           %
%%%%%%%%%%%%%%%%%%%%%%%%%%%%%%%%%%%%%%%%%%%%%%%%%%%%%%%%%%%%%%%%%%%%%%
\title{\vspace{-2em}Chapter 3: Linear maps}
\author{\emph{Linear Algebra Done Right}, by Sheldon Axler}
\date{}

\begin{document}
\maketitle

\newpage
\tableofcontents
\newpage

\section{The vector space of linear maps}

\begin{problem}
Suppose \(b,c \in \R\). Define \(T : \R^3 \to \R^2\) by \[
T(x,y,z) = ( 2x - 4y + 3z + b, 6x + cxyz).\]
Show that $T$ is a linear map if and only if \(b = 0\) and \(c = 0\).
\end{problem}


\begin{proof}
Consider \(T(1,1,1) = T(1,0,0) + T(0,1,1)\)
\begin{align}
  T(1,1,1) &= (1 + b, 6 + c) \\
  T(1,0,0) &= (2 + b, 6) \\
  T(0,1,1) &= (-1 + b, 0)
\end{align}
Therefore, \(b=0, c=0\)
\end{proof}

\section{Null space and Range}

\begin{problem}{1}
Give an example of a linear map $T$ such that \(\dim \Null T = 3\) and \(\dim \Range T = 2\).
\end{problem}

\begin{proof}
Consider \(T : \poly (4) \mapsto \poly (1)\), \(T(a z^4 +b z^3 + c z^2 + d z + e) = (d z + e)\). Then \(\dim \Null T = 3\) and \(\dim \Range T = 2\).
\end{proof}

\begin{problem}{3}
Suppose \(v_1 , \dots, v_m\) is a list of vectors in $V$. Define \(T \in \Hom(\F^m, V)\) by \[
  T(z_1, \dots, z_m) = z_1v_1 + \dots + z_mv_m.\]
\begin{enumerate}[(a)]
  \item What property of $T$ corresponds to \(v_1, \dots, v_m\) spanning $V$.
  \item What property of $T$ corresponds to \(v_1, \dots, v_m\) being linearly independent.
\end{enumerate}
\end{problem}

\begin{proof}
\begin{enumerate}[(a)]
  \item \(\forall v \in V\), we have \(T(z_1, \dots, z_m) = v\), which means \(v_1, \dots, v_m\) span $V$. This suggests that $\Range T$ is equal to $V$. Hence, $T$ is surjective.
  \item If \(v_1, \dots, v_m\) are linearly independent, then \(T(z_1, \dots, z_m) = 0\) implies \(z_1 = \dots = z_m = 0\). This suggests that $\Null T = \{0\}$, hence $T$ is injective.
\end{enumerate}
\end{proof}

\begin{problem}{4}
Show that \[
  {T \in \Hom(\R^5, \R^4) : \dim \Null T > 2}\]
is not a subspace of \(\Hom (\R^5,\R^4)\)
\end{problem}

\begin{proof}
Suppose \(\dim \Null T > 2\), by F.T. of linear maps,
\begin{align}
\dim T = 5 &= \dim \Null T + \dim \Range T \\
&> 2 + \dim \Range T \\
\dim \Range T &< 3
\end{align}

Hence, $T \notin \Hom(\R^5, \R^4)$, and thus not a subspace of \(\Hom (R^5, R^4)\).
\end{proof}

\begin{problem}{5}
Give an example of a linear map \(T : \R^4 \mapsto \R^4\) such that \[
  \Range T = \Null T.\]
\end{problem}

\begin{proof}
Consider \(T : R^4 \mapsto R^4\), \(T(x_1, x_2, x_3, x_4) = (x_3, x_4, 0, 0)\). Then \(\Range T = \Null T\).
\end{proof}

\begin{problem}{6}
Prove that there does not exist a linear map \(T : \R^5 \mapsto \R^5 \) such that \(\Range T = \Null T\).
\end{problem}

\begin{proof}
Suppose there exists a linear map \(T: \R^5 \mapsto \R^5\) such that \(\Range T = \Null T\). Then by F.T. of linear maps, we have \(\dim \Range T = \dim \Null T\), which implies \(\dim \Range T = 5 - \dim \Range T\), or \(\dim \Range T = 2.5\), which is not an integer. Hence, such a linear map does not exist.
\end{proof}

\begin{problem}{7}
Suppose $V$ and $W$ are finite-dimensional with \(2 \leq \dim V \leq \dim W.\) Show that \(\{T \in \Hom(V,W) : T \text{ is not injective }\}\)  is not a subspace of \(\Hom(V,W)\).
\end{problem}

\begin{proof}
$T$ is not injective suggests that \(\dim \Null V >0\).
Then By the F.T. of linear maps, we have
\begin{align}
\dim V &= \dim \Null V + \dim W \\
& \geq 1 + \dim W \geq 1 + \dim V
\end{align}
Contradicts! Hence, \(T \notin \Hom(V,W)\) $\implies$ not a subspace.
\end{proof}

% \begin{problem}{8}
% Suppose $V$ and $W$ are finite-dimensional with \(\dim V \geq \dim W \geq 2.\) Show that \(\{T \in \Hom(V,W) : T \text{ is not surjective }\}\) is not a subspace of \(\Hom(V,W)\).
% \end{problem}

% \begin{proof}

% \end{proof}

\begin{problem}{9}
Suppose \(T \in \Hom (V,W)\) is injective and \(v_1, \dots, v_n\) is linearly independent in $V$. Prove that \(Tv_1, \dots, Tv_n\) is linearly independent in $W$.
\end{problem}

\begin{proof}
$T$ is injective $\implies \Null T = \{0\}$.
Therefore, we have \[
  T(\lambda_1 v_1 + \dots + \lambda_nv_n) \iff \lambda_j = 0\]
where $j \in \{1, \dots, n\}$. Since \(v_1, \dots, v_n\) are linearly independent. It follows that \(Tv_1, \dots, Tv_n\) are linearly independent.
\end{proof}

\begin{problem}{10}
Suppose \(v_1, \dots, v_n\) spans $V$ and \(T \in \Hom(V,W)\). Prove that the list \(Tv_1, \dots, Tv_n\) spans $\Range T$.
\end{problem}

\begin{proof}
Suppose \(v_1, \dots, v_n\) spans $V$. Then \(\forall v \in V\), we have \(v = \lambda_1 v_1 + \dots + \lambda_n v_n\). Then \[
  T(v) = T(\lambda_1 v_1 + \dots + \lambda_n v_n) = \lambda_1 Tv_1 + \dots + \lambda_n Tv_n.\]
This suggests that \(\Range(T) \subset \Span (Tv_1, \dots, Tv_n)\). Also, $\Span (Tv_1, \dots, Tv_n)$ is the smallest containing subspace of $W$ implying that it is a subset of $\Range W$, hence \(\Range T = \Span (Tv_1, \dots, Tv_n)\).
\end{proof}

\begin{problem}{11}
Suppose $S_1, \dots, S_n$ are injective linear maps such that $S_1 S_2 \dots S_n$ makes sense. Prove that \(S_1 S_2 \dots S_n\) is injective.
\end{problem}

\begin{proof}
By F.T. of linear maps, we have
\begin{align}
  \dim S_1 &= \dim \Null S_1 + \dim \Range S_1 \\
  &= 0 + \dim \Range S_1 = \dim \Range S_1
\end{align}
It follows that \[
  \dim S_1 = \dim \Range S_1 = \dim S_2 = \dots = \dim S_n = \dim \Range S_n\]
Hence, $\dim \Null S_1 S_2 \dots S_n = 0 \iff S_1 S_2 \dots S_n$ is injective.
\end{proof}

\begin{problem}{12}
Suppose that $V$ is finite-dimensional and that \(T \in \Hom(V,W)\). Prove that there exists a subspace $U$ of $V$ such that \(\Null T = U\) and \\ \(U \cap \Range T = \{0\}\).
\end{problem}

\begin{proof}
Let $u_1, \dots, u_n$ be a basis for $\Null T$. Then $\Span(u_1, \dots, u_n)$ is a subspace of $V$. Since it is linear independent, it can be extended to a basis of V, say \(u_1, \dots, u_n, v_1, \dots, v_m \). Then $V$ is the direct sum of the spanning of \(u_1, \dots, u_n\) and \(v_1, \dots, v_m\). Take \(U = \Span(v_1, \dots, v_m)\)
\end{proof}

\begin{problem}{13}
Suppose $T$ is a linear map from $\F^4$ to $\F^2$ such that\[
  \Null T = \{(x_1, x_2, x_3, x_4) \in \F^4 : x_1 = 5x_2 \text{ and } x_3 = 7 x_4\}\]
Prove that $T$ is surjective.
\end{problem}

\begin{proof}
A basis of $\Null T$ is \[
  \{(5, 1, 0, 0), (0, 0, 7, 1)\}\]
Then by F.T. of linear maps, we have \[
  \dim \Range T = 4 - \dim \Null T = 4 - 2 = 2\]
Therefore, \(\Range T = \R^2 \implies T\) is surjective.
\end{proof}

\begin{problem}{14}
Suppose $U$ is a 3-dimensional subspace of $\R^8$ and that $T$ is a linear map from $\R^8$ to $\R^5$ such that \(\Null T = U \). Prove that T is surjective.
\end{problem}

\begin{proof}
By F.T. of linear maps, we have \[
  \dim \Range T = 8 - \dim \Null T = 8 - 3 = 5\]
This suggests that \(\Range T = \R^5 \implies T\) is surjective.
\end{proof}

\begin{problem}{15}
Prove that there does not exist a linear map from $\F^5$ to $\F^2$ whose null space equals \[
  \{ (x_1, x_2, x_3, x_4, x_5) \in \F^5 : x_1 = 3 x_2 \text{ and } x_3 = x_4 = x_5\}.\]
\end{problem}

\begin{proof}
Suppose there exists such a linear map, then, by F.T. of linear maps, we have \[
  \dim \Range T = 5 - \dim \Null T = 5 - 2 = 2\]
Contradicts!
\end{proof}

\begin{problem}{16}
Suppose there exists a linear map on $V$ whose null space and range are both finite-dimensional. Prove that $V$ is finite dimensional.
\end{problem}

\begin{proof}
WLOG, let $\dim \Null T = m$ and $\dim \Range T = n$. Then by F.T. of linear maps, we have \[
  \dim V = \dim \Null T + \dim \Range T = m + n < \oo\]
\end{proof}

\begin{problem}{17}
Suppose $V$ and $W$ are both finite-dimensional. Prove that there exists an injective linear map from $V$ to $W$ if and only if \(\dim V \leq \dim W\).
\end{problem}

\begin{proof}
($\implies$) Suppose there is an injective linear map $T: V \mapsto W$. By F.T. of linear maps, we have
\begin{align}
  \dim V &= \dim \Null T + \dim \Range T \\
  &= 0 + \dim \Range T = \dim \Range T
\end{align}

($\impliedby$) Suppose \(\dim V \leq \dim W\). Then there exists a basis of $V$ that can be extended to a basis of $W$. Let $v_1, \dots, v_n$ be a basis of $V$ and $w_1, \dots, w_n$ be a basis of $W$. Define $T: V \mapsto W$ by \(T(v_i) = w_i\). Then $T$ is injective.
\end{proof}

\begin{problem}{19}
Suppose $V$ and $W$ are finite-dimensional and that $U$ is a subspace of $V$. Prove that there exists \[T \in \Hom(V,W)\] such that \(\Null T = U\) if and only if \(\dim U \geq \dim V - \dim W\).
\end{problem}

\begin{proof}
($\implies$) Suppose there exists \(T \in \Hom(V,W)\) such that \(\Null T = U\). Since \(\Range T\) is a subspace of $W$, we have \(\dim Range \leq \dim W\). The rest of the proof follows by the F.T. of linear maps.

($\impliedby$) Suppose \(\dim U \geq \dim V - \dim W\), we have
\begin{align}
  \dim U + \dim W \geq \dim V = \dim \Null T + \dim \Range T
\end{align}
Of course we could find such a $T$.
\end{proof}

\begin{problem}{20}
Suppose $W$ is finite-dimensional and \(T \in \Hom(V,W)\). Prove that $T$ is injective if and only if there exists \(S \in \Hom (W,V)\) such that $ST$ is the identity map on $V$.
\end{problem}

\begin{proof}
($\implies$) T is injective $\iff$ \(\dim \Null T = 0\). Then there exists a basis of $V$ that can be extended to a basis of $W$. Let $v_1, \dots, v_n$ be a basis of $V$ and $w_1, \dots, w_n$ be a basis of $W$. Define $S: W \mapsto V$ by \(S(w_i) = v_i\) and \(T: V \mapsto W\) by \(T(v_k) = w_k\) Then

\begin{align}
  ST(v) &= ST(a_1 v_1 + \dots + a_n v_n) \\
  &= S(a_1 w_1 + \dots + a_n w_n) \\
  &= a_1 v_1 + \dots + a_n v_n \\
  &= v
\end{align}

($\impliedby$) Since $v_k$ is a basis, \(\Null T = \{0\}\). This suggests that $T$ is injective.
\end{proof}

\section{Matrix}

\begin{problem}{1}
Suppose $V$ and $W$ are finite-dimensional and $T \in \mathcal{L}(V,W)$. Show that
with respect to each choice of bases of $V$ and $W$, the matrix of $T$ has at least $\operatorname{dim} \Range T$ nonzero entries.
\end{problem}

\begin{proof}
Follows from the F.T. of linear maps, we have \(\dim \Range T = \dim W - \dim \Null T\).
Since $\Null T$ becomes all the zero entries, therefore
the matrix of $T$ has at least $\dim \Range T$ nonzero entries.
\end{proof}

\begin{problem}{2}
Suppose $D \in \mathcal{L}(\mathcal{P}_3(\mathbb{R}), \mathcal{P}_2(\mathbb{R}))$ is the differentiation map defined by $Dp = p\prime$. Find a basis of $\mathcal{P}_3(\mathbb{R})$
and a basis of $\mathcal{P}_2(\mathbb{R})$ such that the matrix of $D$ with respect to these bases is
\[
  \begin{pmatrix} 1 & 0 & 0 & 1 \\ 0 & 1 & 0 & 0 \\ 0 & 0 & 1 & 0 \\\end{pmatrix}\]
\end{problem}
\begin{proof}
Easy to verify that $\frac{1}{3}x^3, \frac{1}{2}x^2, x, 1$ is a list of basis
of $\mathcal{P}_3(\mathbb{R})$, and its derivative is $x^2, x, 1$ which is a basis of $\mathcal{P}_2(\mathbb{R})$.
\end{proof}

\begin{problem}{3}
Suppose $V$ and $W$ are finite-dimensional and $T \in \mathcal{L}(V,W)$. Prove that there exists a basis of $V$ and a basis of $W$ such that with respect to these bases, all entries of $\mathcal{M}(T)$ are zero except for the entries in row $j$, column $j$, equal 1 for $1 \le j \le \operatorname{dim}\Range T$.
\end{problem}

\begin{proof}
Let $v_1, \ldots , v_n$ be a basis of $V$ such that $\forall i \in {1, \ldots, k}, Tv_i = 1$,
where $k = \operatorname{dim}\Range T$.
Of course, it is a basis of $\Range T$. Expressing this as a matrix gives the desired result.
\end{proof}

\begin{problem}{4}
Suppose $v_1, \ldots ,v_m$ is a basis of $V$ and $W$ is finite-dimensional.
Suppose $T \in \mathcal{L}(V,W)$. Prove that there exists a basis $w_1, \ldots ,w_n$
of $W$ such that all entries in the first column of $\mathcal{M}(T)$ (with respect to
the bases $v_1, \ldots ,v_m$ and $w_1, \ldots ,w_n$) are 0 except for possibly a 1 in
the first column.
\end{problem}
\begin{proof}

\end{proof}

%End
\end{document}
