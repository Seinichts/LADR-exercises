\section{Inner products and Norms}
\begin{exercise}
Show that the function that takes $((x_1,x_2),(y_1,y_2)) \in \mathbb{R}^2 \times \mathbb{R}^2$
to $\left\vert x_1y_1 \right\vert + \left\vert x_2y_2 \right\vert$ is not an inner product on $\mathbb{R}^3$
\end{exercise}
\begin{proof}
	For $((1,1), (1,1)), ((-1.-1), (1,1)) \in \mathbb{R}^2 \times \mathbb{R}^2$, we have \[
		\left\vert 1 \cdot 1 \right\vert + \left\vert 1 \cdot 1 \right\vert = 2 \]
	for both two vectors. But on the same hand, we also have
	\begin{align*}
		((1,1),(1,1)) + ((-1,-1),(1,1)) & = ((0,0),(1,1))                                                           \\
		                                & = \left\vert 0 \cdot 1 \right\vert + \left\vert 0 \cdot 1 \right\vert = 0
	\end{align*}
\end{proof}
This could not be an inner product since it violates
the additivity property of inner products.

\begin{exercise}
Show that the function that takes $((x_1,x_2,x_3),(y_1,y_2,y_3)) \in \mathbb{R}^{3} \times \mathbb{R}^3$
to $x_1y_1 + x_3y_3$ is not an inner product on $\mathbb{R}^3$
\end{exercise}
\begin{proof}
	It take $(0,1,0)$ to zero while $(0,1,0) \neq 0$
\end{proof}

\begin{exercise}
Suppose $\mathbb{F} = \mathbb{R}$ and $V \neq \{0\}$. Replace the positivity condition
(which states that $\langle v, v \rangle \geq 0$ for all $v \in V$) in the definition
of an inner product with the condition that $\langle v, v \rangle > 0$
for some $v \in V$. Show that this new definition of an inner product
does not change the set of functions from $V \times V$ to $\mathbb{R}$
that are inner products on $V$.
\end{exercise}
\begin{proof}
	We show that the two condition are equivalent on the given space.
	Suppose the positivity condition is satisfied, then the new condition
	is obviously true for some $v \in V$. \par
	Now we suppose the new condition is satisfied, then for all $v \in V$, we have \[
		\langle v, v \rangle = \]
\end{proof}
\begin{exercise}
aa
\end{exercise}
