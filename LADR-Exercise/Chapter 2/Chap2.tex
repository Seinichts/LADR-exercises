\documentclass{extarticle}
\sloppy

%%%%%%%%%%%%%%%%%%%%%%%%%%%%%%%%%%%%%%%%%%%%%%%%%%%%%%%%%%%%%%%%%%%%%%
% PACKAGES            																						  %
%%%%%%%%%%%%%%%%%%%%%%%%%%%%%%%%%%%%%%%%%%%%%%%%%%%%%%%%%%%%%%%%%%%%%
\usepackage[10pt]{extsizes}
\usepackage{amsfonts}
\usepackage{amsthm}
\usepackage{amssymb}
\usepackage[shortlabels]{enumitem}
\usepackage{microtype} 
\usepackage{amsmath}
\usepackage{mathtools}
\usepackage{commath}
\usepackage{hyperref}

%%%%%%%%%%%%%%%%%%%%%%%%%%%%%%%%%%%%%%%%%%%%%%%%%%%%%%%%%%%%%%%%%%%%%%
% PROBLEM ENVIRONMENT         																			           %
%%%%%%%%%%%%%%%%%%%%%%%%%%%%%%%%%%%%%%%%%%%%%%%%%%%%%%%%%%%%%%%%%%%%%
\usepackage{tcolorbox}
\tcbuselibrary{theorems, breakable, skins}
\newtcbtheorem{prob}% environment name
              {Problem}% Title text
  {enhanced, % tcolorbox styles
  attach boxed title to top left={xshift = 4mm, yshift=-2mm},
  colback=blue!5, colframe=black, colbacktitle=blue!3, coltitle=black,
  boxed title style={size=small,colframe=gray},
  fonttitle=\bfseries,
  separator sign none
  }%
  {}
\newenvironment{problem}[1]{\begin{prob*}{#1}{}}{\end{prob*}}

%%%%%%%%%%%%%%%%%%%%%%%%%%%%%%%%%%%%%%%%%%%%%%%%%%%%%%%%%%%%%%%%%%%%%%
% THEOREMS/LEMMAS/ETC.         																			  %
%%%%%%%%%%%%%%%%%%%%%%%%%%%%%%%%%%%%%%%%%%%%%%%%%%%%%%%%%%%%%%%%%%%%%%
\newtheorem{thm}{Theorem}
\newtheorem*{thm-non}{Theorem}
\newtheorem{lemma}[thm]{Lemma}
\newtheorem{corollary}[thm]{Corollary}

%%%%%%%%%%%%%%%%%%%%%%%%%%%%%%%%%%%%%%%%%%%%%%%%%%%%%%%%%%%%%%%%%%%%%%
% MY COMMANDS   																						  %
%%%%%%%%%%%%%%%%%%%%%%%%%%%%%%%%%%%%%%%%%%%%%%%%%%%%%%%%%%%%%%%%%%%%%
\newcommand{\Z}{\mathbb{Z}}
\newcommand{\R}{\mathbb{R}}
\newcommand{\C}{\mathbb{C}}
\newcommand{\F}{\mathbb{F}}
\newcommand{\bigO}{\mathcal{O}}
\newcommand{\Real}{\mathcal{Re}}
\newcommand{\poly}{\mathcal{P}}
\newcommand{\mat}{\mathcal{M}}
\DeclareMathOperator{\Span}{span}
\newcommand{\Hom}{\mathcal{L}}
\DeclareMathOperator{\Null}{null}
\DeclareMathOperator{\Range}{range}
\newcommand{\defeq}{\vcentcolon=}
\newcommand{\restr}[1]{|_{#1}}


%%%%%%%%%%%%%%%%%%%%%%%%%%%%%%%%%%%%%%%%%%%%%%%%%%%%%%%%%%%%%%%%%%%%%%
% SECTION NUMBERING																				           %
%%%%%%%%%%%%%%%%%%%%%%%%%%%%%%%%%%%%%%%%%%%%%%%%%%%%%%%%%%%%%%%%%%%%%%
\renewcommand\thesection{\Alph{section}:}



%%%%%%%%%%%%%%%%%%%%%%%%%%%%%%%%%%%%%%%%%%%%%%%%%%%%%%%%%%%%%%%%%%%%%%
% DOCUMENT START              																			           %
%%%%%%%%%%%%%%%%%%%%%%%%%%%%%%%%%%%%%%%%%%%%%%%%%%%%%%%%%%%%%%%%%%%%%%
\title{\vspace{-2em}Chapter 2: Finite dimensional vector space}
\author{\emph{Linear Algebra Done Right}, by Sheldon Axler}
\date{}

\begin{document}
\maketitle

\newpage
\tableofcontents
\newpage



%%%%%%%%%%%%%%%%%%%%%%%%%%%%%%%%%%%%%%%%%%%%%%%%%%%%%%%%%%%%%%%%%%%%%
% SECTION A            																			           
%%%%%%%%%%%%%%%%%%%%%%%%%%%%%%%%%%%%%%%%%%%%%%%%%%%%%%%%%%%%%%%%%%%%%
\section{Span and Linear independence}

\section{Bases}

% Problem 1

\begin{problem}{1}
  Find all vector spaces that have exactly one basis.
\end{problem}

\begin{proof}
\par Consider all lines passing through origin.
\end{proof}

\begin{problem}{3}
\begin{enumerate}[(a)]
\item Let $U$ be the subspace of $\mathbb{R}^5$ defined by
\[ U = {(x_1, x_2, x_3, x_4, x_5) \in \mathbb{R}^5 : x_1  = 3 x_2 \text{ and } x_3 = 7 x_4}\]
Find a basis of $U$.
\item Extend the basis in part (a) to a basis of $\R^5$.
\item Find a subspace $W$ of $\R^5$ such that \( \R^5 = U \oplus W\).
\end{enumerate}
\end{problem}

\begin{proof}
\begin{enumerate}[(a)]
\item We can verify that \((3, 1, 0, 0, 0), ~ (0, 0, 7, 1, 0), ~ (0, 0, 0, 0, 1)\)
is a basis of $U$, and it spans U.
\\ Idea: We start from the basis of $\R^5$ and try to fit it into the given condition.

\item \((3, 1, 0, 0, 0), ~ (0, 0, 7, 1, 0), ~ (0, 0, 0, 0, 1), ~ (0, 1, 0, 0, 0), ~ (0, 0, 1, 0, 0)\)
\\ Idea: Downgrade all vector to one dimensional (to obtain the basis of $\R^5$).

\item \( \Span\{(0, 0, 0, 0, 1), ~ (0, 1, 0, 0, 0)\} \)
\\ Idea: Refer to the definition of direct sum.
\end{enumerate}
\end{proof}

\begin{problem}{5}
  Prove or disprove: there exists a basis \(p_0, p_1, p_2, p_3\) of \(\mathcal{P}_3 (\F)\)
  such that none of the polynomials \(p_0, p_1, p_2, p_3\) has a degree 2.
\end{problem}
\begin{proof}
  Consider
  \begin{align*}
    p_0 &= a_1 \\
    p_1 &= a_2 x \\
    p_2 &= a_3 x^3 \\
    p_3 &= a_4 x^2 + a_5 x^3
  \end{align*}
  It is easy to verify that they are linearly independent, and span $\mathcal{P}_3(\F)$
\end{proof}

\begin{problem}{6}
  Suppose \(v_{1}, v_{2}, v_{3}, v_{4}\) is a basis of $V$. Prove that \[
    v_{1} + v_{2}, v_{2} + v_{3}, v_{3} + v_{4}, v_{4}\]
  is also a basis of $V$.
\end{problem}
\begin{proof}
We could obtain \(v_{1}, v_{2}, v_{3}\) with the following operation
\begin{align}
v_{3} + v_{4} - v_{4} &= v_{3} \\
v_{2} + v_{3} - (v_{3} + v_{4} - v_{4}) &= v_{2} \\
v_{1} + v_{2} - (v_{2} + v_{3} - (v_{3} + v_{4} - v_{4})) &= v_{1}
\end{align}
Which suggests that \(v_{1} + v_{2}, v_{2} + v_{3}, v_{3} + v_{4}, v_{4}\)
also spans $V$. \\
Next, we prove that they are linearly independent.

\begin{align}
& a(v_{1} + v_{2} + b ( v_{2} + v_{3} ) + c (v_{3} + v_{4}) + d v_{4} \\
= & a v_{1} + (a + b) v_{2} + (b + c) v_{3} + (c + d) v_{4}
\end{align}
Since \(v_{1}, v_{2}, v_{3}, v_{4}\) is a basis, we have \(a = b = c = d = 0\).
This implies that \(v_{1} + v_{2}, v_{2} + v_{3}, v_{3} + v_{4}, v_{4}\)
is also linearly independent.
\end{proof}

\begin{problem}{8}
  Suppose $U$ and $W$ are subspaces of $V$ such that \(V = U \oplus W\).
  Suppose also that \(u_{1},\dots,u_{m}\) is a basis of $U$ and \(w_{1},\dots,w_{n}\)
  is a basis of $W$. Prove that \[u_{1},\dots,u_{m}, w_{1},\dots,w_{n}\]
  is a basis of $V$.
\end{problem}
\begin{proof}
  The linear independency of \(u_{1},\dots,u_{m}, w_{1},\dots,w_{n}\) is directly come from the definition of direct sum since \(U \cap W = {0}\).
  Since \(u_{1}, \dots, u_{m}\) is a basis of U. We can write any \(u \in U\) as a linear combination of \(u_{1}, \dots, u_{m}\). Similarly, we can write any \(w \in W\) as a linear combination of \(w_{1}, \dots, w_{n}\).
  Therefore, we can write any \(v \in V\) as a linear combination of \(u_{1}, \dots, u_{m}, w_{1}, \dots, w_{n}\).
\end{proof}
% End

\section{Dimension}

\begin{problem}{1}
Suppose $V$ is finite-dimensional and $U$ is a subspace of $V$ such that $ \dim U = \dim V$. Prove that $U = V$.
\end{problem}

\begin{proof}
Let \(u_1, \dots, u_n\) be a basis of $U$. Since $U$ is a subspace of $V$, we can extend the basis of $U$ to a basis of $V$.
Since \( \dim U = \dim V\), every independent list of right length is also a base suggests that \(u_{1}, \dots, u_{n}\) also spans $V$. Therefore, \(U = V\).
\end{proof}

\begin{problem}{2}
Show that the subspaces of $\R^2$ are precisely the zero subspace, all lines in $R^{2}$ through the origin, and $\R^2$ itself.
\end{problem}
\begin{proof}
\( \dim R^{2} = 2 \implies \dim \text{subspaces of } R^{2} \leq 2\)

\begin{align}
  \text{When} \hspace{10pt} &\dim U = 0,& U  = \set{0}.  \\
  &\dim U = 1,& U \text{ is a line passing through origin.}  \\
  &\dim U = 2,& U = R^{2}.
\end{align}



\end{proof}

% \begin{problem}{4}
% \begin{enumerate}[(a)]
%   \item Let $U = {p \in P_{4}(\R) : p(6) = 0}$. Find a basis of $U$.
%   \item Extend the basis in part (a) to a basis of $P_{4}(\R)$.
%   \item Find a subspace $W$ of $P_{4}(\R)$ such that \(P_{4}(\R) = U \oplus W\).
% \end{enumerate}
% \end{problem}

% \begin{proof}
% Proof
% \end{proof}


\end{document}
